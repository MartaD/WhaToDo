\documentclass[pdflatex,11pt]{aghdpl}

\usepackage[polish]{babel}
\usepackage[utf8]{inputenc}
\usepackage[T1]{fontenc}

% dodatkowe pakiety
\usepackage{enumerate}
\usepackage{pdfpages}
\usepackage{afterpage}
\usepackage{pdflscape}
%\usepackage{rotating}

\graphicspath{{./img/}}
%\usepackage{subfigure} %kilka obrazkow w ramach jednego figure

%---------------------------------------------------------------------------

\author{Marta~Drabarczyk, Krzysztof~Kutt, Michał~Nowak}
\titlePL{WhaToDo. Pomoc w organizacji czasu}
\thesistypePL{Zaawansowane Technologie Bazodanowe}
\date{2012}
\departmentPL{Katedra Automatyki}
\facultyPL{Wydział Elektrotechniki, Automatyki, Informatyki i Elektroniki}
\setlength{\cftsecnumwidth}{10mm}

%---------------------------------------------------------------------------

\begin{document}

\titlepages

\tableofcontents
\clearpage

\chapter{Ogólny opis systemu}

Z podziałem na sekcje, albo po prostu kilka zdań ciągłego tekstu

\section{Cel systemu}

Po co to wszystko?

\section{Użytkownicy}

Dla kogo. Możliwość korzystania przez stronę internetową (w przyszłości również przez przeglądarkę w telefonie).

\section{Zastosowanie}

Wizyta u dentysty. Programiku, powiedz przecie, kiedy iść do niego mogę?


% podział na odpowiednie
% \section{Nazwa}
% i ewentualnie
% \subsection*{Opis produktu}

% rysunki:
%\begin{figure}[!h]
%\centering
%\includegraphics[width=\textwidth]{Prezentacja1}
%\caption{Schemat procesu implementacji}
%\label{schematProcesuImplementacji}
%\end{figure}

% ===========================================================================

\chapter{Problem: Zarządzanie czasem}

\section{Jak zarządzać czasem?}

Tutaj jakiś wstęp z informacjami o różnych podejściach

\section{Nasza metoda}

Metoda, którą wybraliśmy. Kto i kiedy wymyślił? Jak działa i dlaczego? Opisać jakieś przykłady zastosowania - chociażby to, że Michał korzysta i działa ;)


% ===========================================================================

\chapter{Analiza problemu}

\section{Przypadki użycia}

\subsection{Ogólny diagram}

Diagram

\subsection{Definicje przypadków użycia}

\subsubsection*{UC1}

Zerknąć na to co było w prezentacji pokazywane kiedyś.


\section{Baza danych}

Diagram


% ===========================================================================

\chapter{Metodyka pracy w naszej grupie}

Jak nam przebiegała praca w grupie. Z czego korzystaliśmy, żeby się wspierać. Jak dzieliliśmy się zadaniami. Jaka metodyka projektowania. Etc.


% ===========================================================================

\chapter{Implementacja}

\section{Technologia}

Netbeans, Ruby on rails, PostgreSQL, HTML5, ...\\
Przy RoR wspomnieć o tym, że wymusza MVC (chociaż u nas i tak nie jest to w pełni zachowane :P)

\section{Moduły}

Użytkownicy, Taski, Kalendarz, ...

\section{Algorytmy}

Tutaj na pewno trzeba opisać algorytm sugestii

\section{Warstwa Prezentacj}

Tutaj krótki opis i kilka zrzutów ekranu

% ===========================================================================

\chapter{Podsumowanie}

\section{Co udało się zrealizować?}

Czy jesteśmy zadowoleni i takie tam.

\section{Napotkane problemy}

Chociażby to, że Ruby jest dla nas nieznaną technologią.

\section{Możliwości dalszego rozwoju projektu}

Trzeba zaglądnąć w listę rzeczy, które mamy w specyfikacji i jakoś opisać + może mamy jakieś inne pomysły jeszcze ;)


\end{document}

